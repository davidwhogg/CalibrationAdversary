\documentclass{article}
\usepackage[utf8]{inputenc}
\usepackage[letterpaper]{geometry}

\title{\bfseries%
How to transfer a wavelength solution from an arc or laser frequency comb to a science exposure}
\author{David W. Hogg \and Lily L. Zhao}
\date{2025 January}

\begin{document}
\maketitle

\begin{abstract}\noindent
Spectroscopic images are usually wavelength calibrated with arc lamp exposures (ThAr, say) or else laser frequency comb or etalon.
All of these illuminate the spectrograph with emission lines.
These emission lines are often measured with peak-centroiding methods, whereas stellar Doppler shifts are usually measured from absorption features, with methods that look like cross-correlations or maximum-likelihood estimators.
The difference in treatment between the calibration lines and the science exposures generically lead to calibration biases, especially when the line-spread function varies over the spectrograph focal plane.
Here we demonstrate these biases, and deliver recommendations for making unbiased measurements of stellar Doppler shifts with typical data sets.
Fundamentally the message is that it is imperative to measure the calibration sources with the same tools that are used for the science exposures.
\end{abstract}

Hello World.
\end{document}
